\chapter{Introduction}

\section{Background and motivation}
On June 30th 2000, nine young men lost their lies in a crowd crush during a Pearl Jam concert at Roskilde Festival in Denmark \cite{pearl_jam}. An uncontrolled surge, pushing the crowd towards the scene, caused immense pressure on the front most concert-goers, thrusting them against the barriers. The high-energy mass of people unknowingly trampled the victims, who succumbed under the pressure of the crowd. This incident is unfortunately not the only one of its kind, as crowd crushes continue to occur at mass gatherings around the world.

\section{Problem definition}
\label{sec:problem-definition}

No matter the size of the event, crowd safety is a complex and multifaceted problem. No matter the extent of planning and preparation, the unexpected can still happen. Crowd safety professionals have an immense responsibility, as they are tasked with ensuring the safety of thousands of people. Large crowds are often unpredictable, and improper planning and/or response can lead to disastrous consequences. Fortunately, however, crowd safety professionals have a plethora of tools and knowledge at their disposal to help mitigate these risks and keep crowd dynamics within manageable bounds (see section \ref{sec:crowd-safety}).

Despite the available tools and established frameworks, current crowd safety management practices face significant limitations. A recurring theme identified through discussions with industry professionals is a heavy reliance on the experience and intuition of the safety team. Key decisions regarding venue layout, capacity planning, resource allocation (e.g., staffing levels and placement), and risk assessment of concerts often depend heavily on estimations derived from past events and anecdotal knowledge rather than objective, quantitative data specific to the event's context. While experience is invaluable, this reliance introduces subjectivity and potential inconsistencies. Calculations conducted for aspects such as stage layout or entrance/exit dimensions are often based on estimations made by the safety team, albeit these usually are adjusted to cover worst-case scenarios. Even these adjustments, however, are still limited by the imagination and, crucially, the past experiences of the safety professionals. This ultimately suggests that younger or less experienced teams may lack the extensive reference points available to seasoned experts, potentially leading to false assumptions and miscalculations. Consequently, the efficacy of safety planning can appear correlated with the cumulative experience within the team.

This complicated problem, has a seemingly simple and obvious solution: more objective data. At present, experiences and observations often go undocumented, as the safety team already has a plethora of responsibilities and tasks to attend to, not to mention the impossibility of having a full overview of the event at one given moment. Moreover, most music festivals are massively scalable operations, going from a relatively small team of full-time employees and contractors throughout the planning stage, to a massive team of volunteers during the event. In the example of Roskilde Festival, this culminates in a team scaled from over 100 employees to over 30 thousand. This drastic staff scaling certainly applies to the crowd safety department as well, potentially contributing to a lack of continuity and knowledge retention. This implies that immediately following the event, or even after a given day, important observations and learnings from staff may be lost if not systematically documented. These factors effectively hinder a comprehensive, data-driven understanding of crowd behavior.

Furthermore, communicating the rationale behind safety decisions and requirements to other internal departments or external stakeholders can be challenging without clear, objective evidence. Provided that safety precautions are based on subjective assessments, conveying a need for specific resources or precautions can prove difficult. For instance, Roskilde Festival's safety team present a recurring challenge of convincing colleagues in the food and beverage department of concerns regarding the placement of food stalls or bars. Without the aid of clear evidence, Roskilde Festival's safety team occasionally find themselves dedicating valuable time and resources to justify their positions, at times even having to deviate from their core competencies to develop visual material to support their arguments.

These scenarios and limitations highlight opportunities for significant improvement, namely through the development and integration of technology capable of providing objective, measurable insights into crowd dynamics at music festivals. This thesis seeks to explore these opportunities, guided by the following hypotheses:

\begin{itemize}
  \item Challenges in communicating crowd safety requirements and justifying decisions internally are often due to the subjective nature of current assessments, lacking objective, easily understandable evidence.

  \item The precision and efficacy of safety planning are constrained by a reliance on experience-based estimations rather than quantitative, historical data on actual crowd dynamics.

  \item The lack of a persistent, easily accessible digital record of crowd dynamics during an event limits post-event analysis, knowledge retention, and continuous improvement within safety teams.
\end{itemize}

\section{Brief history of Fluxense}

Together with two classmates, I founded a startup, Fluxense, in January 2024. Our initial plan was to solve the crowd safety challenges of music festivals, as presented in section \ref{sec:problem-definition}, by developing an AI-enabled system for monitoring existing CCTV infrastructure to provide automated analyses of crowd behavior. We gained traction quickly, with several large festivals expressing their interest in our proposed product. Development began almost immediately, and we held our first prototype test at DTU's Commemoration Day, where we provided a live count of the number of people in the concert hall. The test gave great results, as well as valuable learnings, and became the first of many. The following summer was very busy, as we attended three of Denmark's largest music festivals -- Copenhell, Roskilde Festival and Smukfest -- to further test and develop our product.

After the summer, we stood at a crossroads. Our collaborations with the different festivals had revealed that each had their own unique requirements, and the value of our product was not as clear-cut as we had initially thought. We feared that crowd safety was not a large enough market for scaling our business, nor that a generalized product would be attractive in the industry. We decided to pivot, and began exploring other markets where our technology could be of use. We gradually moved away from our initial focus on crowd safety, and found business intelligence to be a much larger and lucrative market. Instead of monitoring crowds, our new product would track individual customers in retail stores, transportation hubs, amusement parks, and museums. We aimed to provide insights into places/products of interest, dwell times, conversion rate, footfall analysis, etc., to help businesses optimize their operations.

As the autumn progressed, we began securing new collaborations in our target industry, and our value proposition became clearer. One important thing had been lost in the process, however: our motivation. We had started Fluxense with the goal of improving crowd safety at music festivals, as it was a mission we shared a passion for. Our new focus on business intelligence made sense fiscally, but didn't evoke the same feeling of purpose. Fluxense ended up dissolving in the winter of 2024/25, as we couldn't see ourselves in the startup's new reality, and struggled to find a common vision.

\section{Scope and purpose statement}
\label{sec:mission-statement}

This thesis continues approximately where Fluxense left off before the pivot. However, rather than following the path laid out by the startup and striving to develop a scalable commercial product, the purpose of this work is to create a tool that directly aids crowd safety managers at Roskilde Festival. This is partially due to our already close collaboration throughout the entirety of 2024, as well as their expressed interest in continuing our collaboration through this leg of the project. Additionally, it is the largest music festival in Northern Europe, attracting over 130 thousand guests each year \cite{rf}. With considerable prestige in the industry, as well as a passionate dedication to improving crowd safety, Roskilde Festival is an ideal partner for this project.

It is important to note that the majority of crowd safety practices presented in this thesis are gathered through discussions with Roskilde Festival's safety team. While occasional references may be made to interactions or insights gained through collaborations with other festivals, these are included solely to illustrate the broader landscape and are not indicative of the project's applicability beyond Roskilde. These findings are also assumed limited to a Danish context, as all discussions with crowd safety professionals were with Danish festivals. Additionally, it was observed during these consultations that some prominent experts occasionally presented viewpoints that could be interpreted as subjective or opinionated. However, as the explicit goal of this project is to provide a functional tool for Roskilde Festival's specific operational environment, a critical evaluation of the objective validity of these statements falls outside the defined scope and is not deemed essential for achieving the project's objectives.

In summary, the purpose statement of this project is as follows:
\begin{quote}
  \textit{To enhance Roskilde Festival's crowd safety management by developing an intuitive, data-driven platform that provides actionable insights into crowd dynamics, thereby improving planning, internal communication, and documentation for the safety team.}
\end{quote}

\section{Objectives}
\label{sec:objectives}

Formalizing the hypotheses presented in section \ref{sec:problem-definition}, the business objectives of this project are as follows:
\begin{enumerate}
  \item \textbf{Improve internal communication}: offer clear, visual, objective evidence to help the safety team communicate requirements and justify decisions to other departments.
  \item \textbf{Enhance safety planning}: provide quantitative, historical data on crowd dynamics to enable more accurate planning of layouts, capacities, resource allocation, and facility placement.
  \item \textbf{Create reliable documentation}: Generate a persistent digital record of crowd dynamics for post-event analysis, debriefing, and knowledge retention.
\end{enumerate}

Subsequently, the requirement specifications of the product are as follows:
\begin{enumerate}
  \item \textbf{Core metric extraction}: The product must automatically process inputted data, and accurately extract key anonymized crowd metrics, including density, ingress/egress flow rates, people counts, and movement patterns.

  \item \textbf{Intuitive user interface}: The product must provide a user-friendly web interface featuring interactive visualizations. Users must be able to easily filter data and navigate through the application.

  \item \textbf{Compliant data handling}: The system must securely handle input data, ensuring compliance with legislation such as the European Union's General Data Protection Regulation (GDPR). It must generate anonymized results in order to retain historical data for comparative analysis and planning.
\end{enumerate}


\section{Thesis structure}

In their book, \textit{Design Science}, Hubka and Eder characterize the design process as intuitive, iterative, innovative, unpredictable and reflective \cite{hubka_eder}. While these aspects are inherent to design, tackling complex engineering challenges requires more than intuition and creativity alone. To manage the process effectively, ensure thoroughness, and facilitate clear understanding and traceability, a structured approach is beneficial \cite{eder}. Therefore, the process needs to be organized, drawing upon established product design methodologies and frameworks to guide this project. Many such frameworks exist, offering different levels of detail and focus. This section will explore the most relevant frameworks, and propose a design and development methodology for this project.

\subsection{Comparison of frameworks}
\vspace{2em}
\begin{figure}[H]
  \centering
  \includegraphics[width=\textwidth]{Pictures/Figures/cross.png}
  \caption{Cross' four-stage model of the design process}
  \label{fig:cross}
\end{figure}

Cross \cite{cross} proposes likely the most simplistic, yet well-known framework: a four-stage model comprised of \textit{exploration}, \textit{generation}, \textit{evaluation} and \textit{communication} (Figure \ref{fig:cross}). Cross describes this type of modal as descriptive, as it merely attempts to model the conventional, heuristic design process. More detailed models of this type exist, such as French's \cite{french} "anatomy of design," (Figure \ref{fig:french}) detailing four stages, most distinctly underlining the problem analysis and definition, as conducted in section \ref{sec:problem-definition}. According to Cross, these models differ from prescriptive models, which offer a more systematic procedure, as well an emphasis on analyzing and understanding the design problem before generating solution concepts. Perhaps the most well-known of these is offered by Pahl et. al \cite{pahl_beitz}, and is based on the following design stages: \textit{clarification of the task}, \textit{conceptual design}, \textit{embodiment design}, and \textit{detail design}. Combining the aforementioned models, Ulrich and Eppinger present a rather comprehensive framework. Their process is based on the following stages: \textit{concept development}, \textit{system-level design}, \textit{detail design}, and \textit{testing and refinement} \cite{ulrich_eppinger}.

\begin{figure}[H]
  \centering
  \includegraphics[width=4cm]{Pictures/Figures/french.png}
  \caption{A block diagram illustrating the design process according to French. The circles represent stages reached, and the rectangles represent work in progress.}
  \label{fig:french}
\end{figure}

These frameworks provide varying degrees of structure and granularity to the design/development process, but all share the commonality of being highly engineering-focused. In engineering a physical product, a rigid, structured process is often necessary as each iteration must be designed, manufactured and tested. This is costly, both in effort and material costs. Therefore, the design and development process are divided and sequential. Software, on the other hand, is much more flexible, with iterations being a magnitude faster and cheaper to develop. This demonstrates a need for adapting the design/development process to the context of the product being developed. Conveniently, Ulrich and Eppinger present a multitude of adaptations to their framework, including what they refer to as "Quick-Build Products" and "Digital Products." Here the \textit{detail design} and \textit{testing and refinement} stages are omitted, and replaced with a cyclical design-build-test process. Whereas the linear, rigid processes described previously are labelled as "waterfall methods", this iterative process is most often referred to as \textit{agile development}.

Agile development has many benefits in contrast to the waterfall approach, especially in the context of software development. As mentioned previously, the waterfall approach is ideal for engineering projects where prototyping is costly. When the cost of prototyping is negligible, however, agile methodology grants the flexibility to iterate quickly, and adapt to evolving requirements. Design and development are sequential in a waterfall model; here they are heavily intertwined. A strong example of this, as well as being the most popular implementation of agile development, is \textit{Scrum}. Scrum defines the following stages: \textit{sprint planning}, \textit{daily stand-up}, \textit{sprint review}, and \textit{sprint retrospective}, with a sprint typically lasting 2-4 weeks \cite{scrum}. This framework is ideal for large teams, as it ensures that all team members are aligned, and are able to coordinate their efforts efficiently. The daily stand-up is a particularly useful tool for larger teams, preventing overlapping work, or potential blockers from being overlooked. In smaller teams, however, this structure can be cumbersome, and potentially even counterproductive. Especially when considering this project, exploring a singular use-case as a solitary developer, the full-scale implementation of Scrum is evidently not necessary. Instead, a more adaptable and lightweight framework is employed.

In his book, \textit{The Lean Startup}, Eric Ries describes a simple, yet effective agile framework, which he refers to as the \textit{Build-Measure-Learn} loop \cite{lean_startup}. This framework is designed for rapid prototyping and iteration, splitting each stage into \textit{build}, \textit{measure} and \textit{learn}. \textit{Build} involves developing a minimal product or feature, which is then tested with the target user(s) in the \textit{measure} stage. The results of this test are then studied in the \textit{learn} stage, where the product/feature is adapted based on these results. This process is then repeated, until the requirements are met. This framework is ideal for this project, as it is quite exploratory in nature, while still aiming to fulfill predetermined requirements.

\subsection{Design and development methodology}

Building upon the exploration of frameworks conducted above, the following design methodology is proposed for this project. The design stage follows an engineering approach, guided by Ulrich and Eppinger's \textit{concept development} and \textit{system-level design} stages. Subsequently, the development stage is based on a more agile, entrepreneurial approach, as described by Ries. Standing in for the \textit{detail design} and \textit{testing and refinement} stages, Ries's \textit{Build-Measure-Learn} architecture is employed in order to facilitate rapid prototyping and iteration, albeit with a slight augmentation. The \textit{measure} and \textit{learn} stages are consolidated through periodic feedback sessions with Roskilde Festival, where implemented features are reviewed and emerging requirements are gathered. This eliminates the implied analysis between these two stages, as this project is developed in close collaboration with the target user, and its usage is not intended for a wider audience.

Chapter \ref{chap:concept-development}, \textbf{Concept Development}, ultimately selects and proposes a conceptual solution to the problem outlined in section \ref{sec:problem-definition}. As defined by Ulrich and Eppinger, a concept is "a description of the form, function, and features of a product and is usually accompanied by a set of specifications, an analysis of competitive products, and an economic justification of the project." \cite{ulrich_eppinger} This selection is initially preceded by a thorough inspection into the intricacies of crowd safety management, as well as a review of potential solutions and existing products. A novel solution is thereafter presented, outlining requirement specifications, as well as its technical, legal, and financial feasibility.

Chapter \ref{chap:system-level-design} focuses on the \textbf{System-level Design}, including the definition of the product architecture, and a decomposition of the product into subsystems and components. The full workflow from data collection to the resulting user interface is presented, followed by a detailed description of each sub-system, including the data collection, computer vision model, spatial mapping, metric extraction, and the user interface/frontend.

Finally, Chapter \ref{chap:results} presents the \textbf{Results} of the development stage, beginning with a showcase of the frontend, including an overview of the iterative feature selection conducted in accordance with the augmented \textit{Build-Measure-Learn} framework. This is followed by a technical performance evaluation, describing the accuracy of the solution. The chapter concludes by revisiting the business objectives outlined in section \ref{sec:objectives}, and evaluating the product's business value. This includes a summary of a workshop conducted with Roskilde Festival, where the final product was presented and tested by members of the safety team.