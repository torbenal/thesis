\chapter{Conclusion}

\section{Summary of results}

Chapter \ref{chap:results} showcases the developed frontend application, which provides an intuitive interface for the Roskilde Festival safety team to access and analyze crowd dynamics data through interactive charts and maps. This interface was developed iteratively based on user feedback, directly addressing requirements such as core metric extraction, an intuitive user interface, and compliant data handling that ensures anonymized historical data retention. The chapter also details a technical performance evaluation of the system.

% todo ^^^


Furthermore, the business value of the product was evaluated through a workshop with key members of the Roskilde Festival safety team, where scenarios were used to demonstrate the tool's ability to provide objective data, enhance safety planning, improve internal communication with visual evidence, and create reliable documentation for post-event analysis and knowledge retention. The workshop highlighted the tool's value in replacing subjective estimations with concrete data, thereby increasing confidence in planning decisions and saving time, particularly in communicating safety requirements to other departments. The safety team also identified emerging values, such as its potential role in planning entrances around the Arena stage and determining fencing layouts by providing objective data to evaluate the effectiveness of crowd distribution strategies.

\section{Market expansion opportunities}

\section{Technical challenges and lessons learned}

\section{Technical improvements}

\section{Closing remarks}