\chapter{Conclusion}

This project set out to improve crowd safety measurability at Roskilde Festival by developing an AI-enabled video surveillance analysis platform. The research explored existing crowd safety management practices, identified key limitations in objective data acquisition, and proposed a technical solution to address these gaps.

The core of the project involved designing and implementing a system capable of processing video footage to extract actionable crowd dynamics metrics. This was achieved through a multi-stage process encompassing data collection using deployed cameras, development and fine-tuning of YOLO-based computer vision models for head detection and tracking, spatial mapping via homography to translate pixel coordinates to real-world GIS data, and subsequent extraction of key metrics such as ingress/egress counts, flow rates, cumulative population, crowd density, and movement patterns.

The developed system successfully met its defined objectives, proved through a workshop conducted with key members of the Roskilde Festival safety team.  Participants rated the tool highly for perceived value, effort reduction, and confidence enhancement in addressing typical planning and communication scenarios. The feedback highlighted the tool's potential to streamline workflows, particularly in justifying safety requirements and in planning based on objective data rather than relying on estimations or anecdotal knowledge.

The technical performance evaluation of the fine-tuned head detection models showed an average F1-score of 77.45\% across various camera deployments, with individual models achieving F1-scores as high as 88.11\% (Arena CAM2). This indicates an acceptable level of accuracy in the core detection task, especially in comparison to other methods (Section \ref{sec:technical-solutions}).

The system was designed with data compliance in mind, ensuring that video footage is processed to output anonymized, aggregated data for storage and presentation, adhering to data protection regulations.

In summary, the project successfully developed and validated a functional prototype of an AI-driven platform that provides objective, measurable insights into crowd dynamics, demonstrating a clear value for Roskilde Festival's safety team.

\section{Market expansion opportunities}

While this thesis was specifically tailored to the needs of Roskilde Festival and not intended for commercial use, the underlying technology and the insights gained have significant potential for market expansion beyond this initial application. This project has essentially created a template for other safety-focused event managers. With minor adjustments, the developed system can be adapted to other music festivals, sporting events, or even public gatherings. This technology can also be applied to other sectors, such as transportation hubs (airports, train stations), urban planning (monitoring pedestrian traffic in city centers), or business intelligence (analyzing customer flow in retail environments), as mentioned in Section \ref{sec:fluxense}.

\section{Technical improvements}

Several areas for improvement for the developed system were identified throughout the project. Firstly, the system's performance could be enhanced by improving the camera hardware. The employed cameras were not ideal for low-light conditions, which limited the system's ability to capture and process footage during nighttime hours. This is a significant limitation, as the biggest attractions are typically scheduled last for each day of the festival (typically around 02:00). Utilizing cameras with more consistent switching to infrared (IR) mode or employing models capable of processing low-light footage would address this issue. Secondly, the spatial mapping algorithm could be refined. The proposed method required manual selection of vertices on both the camera image and the GIS map, which potentially introduced human error and minor inaccuracies to the mappings. Finally, the computer vision model could be improved further. While the accuracy of detecting individuals was superior to other methods (Section \ref{sec:technical-solutions}), the models missed approximately 20\% of people in front of the cameras on average. This could be addressed by increasing the size of the training dataset or exploring alternative models.


\section{Closing remarks}
I would again like to express my gratitude to Morten and Mads Therkildsen for their support and interest throughout the past one and a half years. Through my time working with them, I have gained a much deeper understanding of the important work they do, and have gained respect for crowd safety managers in general. I would also like to articulate my appreciation for my supervisor, Thomas J. Howard, for his guidance and support throughout this project, as well as throughout the lifetime of my startup, Fluxense.