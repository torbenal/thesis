\chapter{Concept Development}
\label{chap:concept-development}

\section{Revisiting the problem definition}



\section{Understanding crowd safety management}
\label{sec:crowd-safety}

In order to develop a solution that supports crowd safety professionals, it is imperative to understand how they operate, and what tools they currently have at their disposal. Together with my co-founders at Fluxense, we conducted interviews with many crowd safety professionals from various different organizations, including Event Safety (Smukfest), smash! bang! pow! (Syd for Solen), Roskilde Festival, and Live Nation (Copenhell, Heartland). Throughout this period, it became clearer that crowd safety management is very complex, and is almost as much a philosophy as it is a science. Music festivals and events vary greatly in size, participant demographics, venues, and budget. Equally varied are the crowd safety professionals themselves, which seemed to have varying levels of experience, as well as distinct approaches to their work.

Most interestingly, the greatest discrepancy is seemingly between a focus on incident-prevention and incident-response, or "crowd safety vs. security", as according to Roskilde Festival's Director of Safety, Morten Therkildsen. A security-focused approach often involves less planning, as well as hiring third-party professionals to handle safety during the event. Safety-focused teams, on the other hand, spend most of the year leading up to their events meticulously planning initiatives to ensure the well-being and enjoyment of their guests. The distinction between these two protocols was apparent throughout Fluxense's collaborations with both Copenhell and Roskilde Festival. Live Nation had two full-time employees responsible for crowd safety at Copenhell, whereas Roskilde Festival had a team of 10+ full-time employees.

\subsection{Existing frameworks and workflows}

\subsection{Key metrics}



\section{Comparing technical solutions}

\subsection{GPS}
\subsection{Bluetooth beams}
\subsection{Other camera solutions}
(competitor analysis)

\section{Proposed solution}
\section{Feasibility of solution}
\subsection{Technical feasibility}
\subsection{Legal feasibility}
\subsection{Financial feasibility}


